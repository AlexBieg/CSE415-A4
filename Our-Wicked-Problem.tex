\title{Modeling the Spread of Disease in Africa}
\author{
        Alex Bieg \\
        Cole Chamberlin \\
        CSE 415, University of Washington\\
}
\date{\today}

\documentclass[12pt]{article}

\begin{document}
\maketitle

\section*{Introduction}
Disease is one of humanities biggest setbacks, and luckily, modern medicine
has revolutionized the way we address it. Unfortunately, many countries still
have limited access to such resources, and disease is often rampant among their
populations. Countries like the US commonly offer aid in the form of
on-the-ground support and money. The goal of our project was to model the
distribution of aid as a formulation of state-space search using a heuristic
function to evaluate effectiveness of distribution strategies. 

\section*{State Space}
The problem is modeled as a chronological progression. Each state exists at a
specific time offset from the initial state, and each transition respresents a
constant time increment between starting and ending states. The operators on
each state represent a distribution of aid to specific cities via a third party.
\section*{The Model}
There are several attributes of each state that dictate how the operator
transforms it. Within a state, each city has a population partitioned into
susceptible, infected, and recovered. This is based on common epidemiology
models that use these three quantities to characterize a population. 

\section*{Previous work}

\section*{Results}
In this section we describe the results.

\section*{Conclusions}
We worked hard, and achieved very little.

\end{document}

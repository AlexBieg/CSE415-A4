\title{Modeling the Spread of Disease in Africa}
\author{
        Alex Bieg \\
        Cole Chamberlin \\
        CSE 415, University of Washington\\
}
\date{\today}

\documentclass[12pt]{article}

\begin{document}
\maketitle

\section*{Introduction}
Disease is one of humanities biggest setbacks, and luckily, modern medicine
has revolutionized the way we address it. Unfortunately, many countries still
have limited access to such resources, and disease is often rampant among their
populations. Countries like the US commonly offer aid in the form of
on-the-ground support and money. The goal of our project was to model the
distribution of aid as a formulation of state-space search using a heuristic
function to evaluate effectiveness of distribution strategies. 

\section*{State Space}
The problem is modeled as a chronological progression. Each state exists at a
specific time offset from the initial state, and each transition respresents a
constant time increment between starting and ending states. The operators on
each state represent a distribution of aid to specific cities via a third party.

In our problem formulation we are looking at the 15 largest cities in Africa.
Aid can only ever be distributed to a single city per operator. This means
that we have a branching factor of 15.

One might naiively believe that we should set the goal state to a world where the
disease no longer exists. Unfortunately, Given the nature of most diseases, complete eradication
is not an option. Because of this, we are not including a goal state.

\section*{The Model}
There are several attributes of each state that dictate how the operator
transforms it. Within a state, each city has a population partitioned into
susceptible, infected, and recovered. This is based on common epidemiology
models that use these three quantities to characterize a population.

When aid is applied to a city the rate at which susceptable become infected
will decrease and the rate that infected become recovered will increase. In
addition to the aid applied to a city, it's other features will affect the
attributes.

Citites that are connected to a city can also spread infections between them.
This is affected by the euclidean distance between the cities, the total population
of the two cities and number of airports that each city has.

\section*{Previous work}
We did not create our model in a vacuum and as such it is important to recognize
previous work that came before us. The Mathematical Association of America has
developed a differential model of disease that we adapted for our model. It specifically
denotes the three populations (susceptable, infected, and recovered) that we used. It
also provided the "Infection Equation" which allowed us to effectively model the population, and integrations with the
autograder. of a city.

\section*{Results}
In this section we describe the results.

\section*{Retrospective}
Cole's contributions include, but are not limited to: creating most of the starter code,
the infection spreading algorithm, and developing the city class.

Alex's contributions include, but are not limited to: writing the PDF, the state class
development, hashing and equals functions for the classes, and integrations with the
autograder.

\section*{Conclusions}
We worked hard, and achieved very little.

\end{document}
